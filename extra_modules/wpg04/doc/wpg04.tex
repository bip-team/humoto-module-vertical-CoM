\input{../../../doc/latex/Styles/wpg_notation_define.tex}
\chapter{WPG v.04}

This version of the pattern generator is based on triple integrator with
piece-wise constant \acs{CoP} velocity. The control input is position of the
\acs{CoP}. Apart from that it is very similar to the \acs{WPG} proposed in
\cite{Herdt2010auro}, for more information refer to \cite{Sherikov2016phd}.


%%%%%%%%%%%%%%%%%%%%%%%%%%%%%%%%%%%%%%%%%%%%%%%%%%%%%%%%%%%%%%%%%%%%%%%%%%%%%%%%%%%%%%%%%%%%%%%%%%%%%%
\section{Variables}

%%%%%%%%%%%%%%%%%%%%%%%%%%%%%%%%%%%%%
\subsection{Footstep positions}
Any preview horizon contains $1$ fixed and $0$ or more variable footstep
positions.  Let $M$ be the number of variable steps in the preview horizon,
$\FRAME{p_j}$ -- frame fixed to $j$-th footstep with $j = 0 \dots M$ ($0$-th
footstep is fixed), $\fp_j = (p^x_j, p^y_j)$ -- position of the $j$-th footstep
on the ground plane.


Coordinate transformation matrix from the frame fixed to the $j$-th footstep to
the global frame is defined as
%
\begin{equation}
    \M[p_j][]{R} =
    \begin{bmatrix}
        \cos{\theta_j}  &   -\sin{\theta_j} \\
        \sin{\theta_j}  &   \cos{\theta_j} \\
    \end{bmatrix}
\end{equation}
%
Orientations $\theta_j$ of the footsteps are predetermined to avoid
nolinearity. Position of the $j$-th footstep in the global frame can be found
as
%
\begin{equation}
    \fp_j = \fp_0 + \sum_{i=1}^{j}  \M[p_{i-1}][]{R} \fp[p_{i-1}]_j,
\end{equation}
%
which leads to
%
\begin{equation}
    \begin{bmatrix}
        \fp_0\\
        \fp_1\\
        \vdots\\
        \fp_M\\
    \end{bmatrix}
    =
    \V{1}_M
    \kron
%    \begin{bmatrix}
%        \M{I}\\
%        \M{I}\\
%        \vdots\\
%        \M{I}\\
%    \end{bmatrix}
    \fp_0
    +
    \begin{bmatrix}
        \M{0}           & \M{0}             & \dots & \M{0}             \\
        \M[p_{0}][]{R}  & \M{0}             & \dots & \M{0}             \\
        \vdots          & \vdots            & \dots & \vdots            \\
        \M[p_{0}][]{R}  & \M[p_{1}][]{R}    & \dots & \M[p_{M-1}][]{R}  \\
    \end{bmatrix}
    \begin{bmatrix}
        \fp[p_0]_1\\
        \vdots\\
        \fp[p_{M-1}]_M\\
    \end{bmatrix}.
\end{equation}
%
It is preferable to express variable footstep positions in a frame fixed to the
preceding footstep in order to impose simple bounds on variable footstep
positions.


The number of sampling intervals in the preview horizon is denoted by $N \ge
M$. The footstep position corresponding to $k$-th ($k = 1,\dots,N$) sampling
interval is denoted as $\fph_k = (\hat{p}^x_k, \hat{p}^y_k)$. These positions
can be found using a selection matrices as
%
\begin{equation}
\begin{split}
    \underbrace{
    \begin{bmatrix}
        \fph_1 \\
        \vdots \\
        \fph_N
    \end{bmatrix}
    }_{\FPh}
    & =
    \underbrace{
    \begin{bmatrix}
        \M{I}   &  \M{0}   &   \dots   &   \M{0} \\
        \M{I}   &  \M{0}   &   \dots   &   \M{0} \\
        \M{0}   &  \M{I}   &   \dots   &   \M{0} \\
        \M{0}   &  \M{I}   &   \dots   &   \M{0} \\
        \M{0}   &  \M{I}   &   \dots   &   \M{0} \\
        \vdots  &  \vdots  &   \ddots  &   \vdots \\
        \M{0}   &  \M{0}   &   \dots   &   \M{I} \\
    \end{bmatrix}
    }_{\M{I}_{\mathit{fps}}}
    \begin{bmatrix}
        \fp_0 \\
        \fp_1 \\
        \vdots\\
        \fp_M \\
    \end{bmatrix}
    =
    \begin{bmatrix}
        \M{I}   \\
        \M{I}   \\
        \M{0}   \\
        \M{0}   \\
        \M{0}   \\
        \vdots  \\
        \M{0}   \\
    \end{bmatrix}
    \fp_0
    +
    \begin{bmatrix}
        \M{0}   &   \dots   &   \M{0} \\
        \M{0}   &   \dots   &   \M{0} \\
        \M{I}   &   \dots   &   \M{0} \\
        \M{I}   &   \dots   &   \M{0} \\
        \M{I}   &   \dots   &   \M{0} \\
        \vdots  &   \ddots  &   \vdots \\
        \M{0}   &   \dots   &   \M{I} \\
    \end{bmatrix}
    \begin{bmatrix}
        \fp_1 \\
        \vdots\\
        \fp_M\\
    \end{bmatrix}\\
    & =
%    \underbrace{
%    \begin{bmatrix}
%        \M{I}   \\
%        \M{I}   \\
%        \M{I}   \\
%        \M{I}   \\
%        \M{I}   \\
%        \vdots  \\
%        \M{I}   \\
%    \end{bmatrix}
%    }_{\M{V}_0}
    \underbrace{
        \V{1}_N
        \kron
        \fp_0
    }_{\V{V}_0}
    +
    \underbrace{
    \begin{bmatrix}
        \M{0}   &   \dots   &   \M{0} \\
        \M{0}   &   \dots   &   \M{0} \\
        \M{I}   &   \dots   &   \M{0} \\
        \M{I}   &   \dots   &   \M{0} \\
        \M{I}   &   \dots   &   \M{0} \\
        \vdots  &   \ddots  &   \vdots \\
        \M{0}   &   \dots   &   \M{I} \\
    \end{bmatrix}
    \begin{bmatrix}
        \M[p_{0}][]{R}  & \M{0}             & \dots & \M{0}             \\
        \vdots          & \vdots            & \dots & \vdots            \\
        \M[p_{0}][]{R}  & \M[p_{1}][]{R}    & \dots & \M[p_{M-1}][]{R}  \\
    \end{bmatrix}
    }_{\M{V}}
    \underbrace{
    \begin{bmatrix}
        \fp[p_0]_1\\
        \vdots\\
        \fp[p_{M-1}]_M\\
    \end{bmatrix}
    }_{\FP},
\end{split}
\end{equation}
%
where $\M{V}$ has the following structure
%
\begin{equation}
    \M{V}
    =
    \begin{bmatrix}
        \M{0}           &   \dots   &   \M{0} \\
        \M{0}           &   \dots   &   \M{0} \\
        \M[p_{0}][]{R}  &   \dots   &   \M{0} \\
        \M[p_{0}][]{R}  &   \dots   &   \M{0} \\
        \M[p_{0}][]{R}  &   \dots   &   \M{0} \\
        \vdots          &   \ddots  &   \vdots \\
        \M[p_{0}][]{R}  &   \dots   &   \M[p_{M-1}][]{R} \\
    \end{bmatrix}
    =
    \underbrace{
        \begin{bmatrix}
            \M{0}   &   \dots   &   \M{0} \\
            \M{0}   &   \dots   &   \M{0} \\
            \M{I}   &   \dots   &   \M{0} \\
            \M{I}   &   \dots   &   \M{0} \\
            \M{I}   &   \dots   &   \M{0} \\
            \vdots  &   \ddots  &   \vdots \\
            \M{I}   &   \dots   &   \M{I} \\
        \end{bmatrix}
    }_{\M{I}_R \kron \M{I}}
    \diag{k = 0...M-1}{ \M[p_{k}][]{R} }
\end{equation}
%
where
%
\begin{equation}
    \M{I}_R
    =
    \begin{bmatrix}
        0   &   \dots   &   0 \\
        0   &   \dots   &   0 \\
        1   &   \dots   &   0 \\
        1   &   \dots   &   0 \\
        1   &   \dots   &   0 \\
        \vdots  &   \ddots  &   \vdots \\
        1   &   \dots   &   1 \\
    \end{bmatrix}.
\end{equation}


%%%%%%%%%%%%%%%%%%%%%%%%%%%%%%%%%%%%%
\subsection{Positions of the CoP}
Position of the center of pressure in the end of $k$-th ($k = 1,\dots,N$)
sampling interval is denoted as $\cop_k = (z^x_k, z^y_k)$. The current
\acs{CoP} position is $\cop_0$. In order to obtain simple bounds on the
\acs{CoP} positions instead of general constraints, their positions are
expressed in the local frames fixed to the respective feet, \emph{i.e.}
%
\begin{equation}
    \cop_k = \fph_k + \M[\hat{p}_k][]{R} \cop[\hat{p}_k]_k.
\end{equation}
%
Orientation matrices $\M[\hat{p}_k][]{R}$ can also be selected with
$\M{I}_{\mathit{fps}}$
%
\begin{equation}
    \begin{bmatrix}
        \M[\hat{p}_1][]{R}\\
        \vdots\\
        \M[\hat{p}_N][]{R}\\
    \end{bmatrix}
    =
    \M{I}_{\mathit{fps}}
    \begin{bmatrix}
        \M[p_0][]{R}\\
        \vdots\\
        \M[p_M][]{R}\\
    \end{bmatrix}
\end{equation}
%
All positions of the \acs{CoP} within the preview horizon are
%
\begin{equation}
    \begin{split}
        \begin{bmatrix}
            \cop_1 \\
            \vdots \\
            \cop_N \\
        \end{bmatrix}
        & =
        \FPh
        +
        \diag{k = 1 \dots N}{\M[\hat{p}_k][]{R}}
        \underbrace{
        \begin{bmatrix}
            \cop[\hat{p}_1]_1 \\
            \vdots \\
            \cop[\hat{p}_N]_N \\
        \end{bmatrix}
        }_{\V{Z}}
        =
        \M{V}_0
        +
        \M{V} \FP
        +
        \diag{k = 1 \dots N}{\M[\hat{p}_k][]{R}}
        \CoP\\
        & =
        \V{1}_N
        \kron
        \fp_0
        +
        \left(
            \M{I}_R \kron \M{I}
        \right)
        \diag{k = 0...M-1}{ \M[p_{k}][]{R} }
        \FP
        +
        \diag{k = 1 \dots N}{\M[\hat{p}_k][]{R}}
        \CoP
    \end{split}
\end{equation}
%


%%%%%%%%%%%%%%%%%%%%%%%%%%%%%%%%%%%%%%%%%%%%%%%%%%%%%%%%%%%%%%%%%%%%%%%%%%%%%%%%%%%%%%%%%%%%%%%%%%%%%%
\section{Model of the system}

The model has the following form
%
\begin{align}
    \cstate_{k+1} =& \M{A} \cstate_k + \M{B} \cop_{k+1}\\
    \dcop_k =& \M{D} \cstate_k + \M{E} \cop_{k+1}.
\end{align}
%

After condensing we obtain
%
\begin{align}
    \begin{bmatrix}
        \cstate_1 \\
        \vdots\\
        \cstate_{N}\\
    \end{bmatrix}
    =&
    \M{U}_x \cstate_0
    +
    \M{U}_u
    \begin{bmatrix}
        \cop_1 \\
        \vdots \\
        \cop_N \\
    \end{bmatrix} \\
%
    \underbrace{
    \begin{bmatrix}
        \dcop_0 \\
        \vdots\\
        \dcop_{N-1}\\
    \end{bmatrix}
    }_{\dCoP}
    =&
    \M{O}_x \cstate_0
    +
    \M{O}_u
    \begin{bmatrix}
        \cop_1 \\
        \vdots \\
        \cop_N \\
    \end{bmatrix} \\
\end{align}
%
or
%
\begin{align}
    \cState
    =&
    \M{U}_x \cstate_0
    +
    \M{U}_u
    \left(
        \M{V}_0
        +
        \M{V} \FP
        +
        \diag{k = 1 \dots N}{\M[\hat{p}_k][]{R}}
        \CoP
    \right)\\
%
    \dCoP
    =&
    \M{O}_x \cstate_0
    +
    \M{O}_u
    \left(
        \M{V}_0
        +
        \M{V} \FP
        +
        \diag{k = 1 \dots N}{\M[\hat{p}_k][]{R}}
        \CoP
    \right)\\
\end{align}
%

The unknowns $\CoP$ and $\FP$ can be grouped together
%
\begin{align}
    \cState
    =&
    \underbrace{
    \begin{bmatrix}
        \M{U}_u \diag{k = 1 \dots N}{\M[\hat{p}_k][]{R}}   &   \M{U}_u \M{V} \\
    \end{bmatrix}
    }_{\M{S}}
    \underbrace{
    \begin{bmatrix}
        \CoP\\
        \FP\\
    \end{bmatrix}
    }_{\V{X}}
    +
    \underbrace{
    \M{U}_x \cstate_0
    +
    \M{U}_u \M{V}_0
    }_{\V{s}}\\
%
    \dCoP
    =&
    \underbrace{
    \begin{bmatrix}
        \M{O}_u \diag{k = 1 \dots N}{\M[\hat{p}_k][]{R}}   &   \M{O}_u \M{V} \\
    \end{bmatrix}
    }_{\M{S}_{\dot{z}}}
    \underbrace{
    \begin{bmatrix}
        \CoP\\
        \FP\\
    \end{bmatrix}
    }_{\V{X}}
    +
    \underbrace{
    \M{O}_x \cstate_0
    +
    \M{O}_u \M{V}_0
    }_{\V{s}_{\dot{z}}}\\
\end{align}
%
or
%
\begin{align}
    \cState
    =&
    \begin{bmatrix}
        \left(\M{I} \bkron \M{U}_u^x \right) \diag{k = 1 \dots N}{\M[\hat{p}_k][]{R}}
        &
        \left(\M{I} \bkron \M{U}_u^x \right)
        \left(
            \M{I}_R \kron \M{I}
        \right)
        \diag{k = 0...M-1}{ \M[p_{k}][]{R} }\\
    \end{bmatrix}
    \V{X}
    \\
    &
    +
    \left(\M{I} \bkron \M{U}_x^x \right)
    \cstate_0
    +
    \left(\M{I} \bkron \M{U}_u^x \right)
    \left(\V{1}_N \kron \fp_0 \right)
    \\
    =&
    \begin{bmatrix}
        \left(\M{I} \bkron \M{U}_u^x \right) \diag{k = 1 \dots N}{\M[\hat{p}_k][]{R}}
        &
        \left(
            \M{I}
            \bkron
            \left(
                \M{U}_u^x
                \M{I}_R
            \right)
        \right)
        \diag{k = 0...M-1}{ \M[p_{k}][]{R} }\\
    \end{bmatrix}
    \V{X}
    \\
    &
    +
    \left(\M{I} \bkron \M{U}_x^x \right)
    \cstate_0
    +
    \left(
        \M{I}
        \bkron
        \left(
            \M{U}_u^x
            \V{1}_N
        \right)
    \right)
    \fp_0
    \\
\end{align}
%
and
%
\begin{align}
    \dCoP
    =&
    \begin{bmatrix}
        \left(\M{I} \bkron \M{O}_u^x \right) \diag{k = 1 \dots N}{\M[\hat{p}_k][]{R}}
        &
        \left(\M{I} \bkron \M{O}_u^x \right)
        \left(
            \M{I}_R \kron \M{I}
        \right)
        \diag{k = 0...M-1}{ \M[p_{k}][]{R} }\\
    \end{bmatrix}
    \V{X}
    \\
    &
    +
    \left(\M{I} \bkron \M{U}_x^x \right) \cstate_0
    +
    \left(\M{I} \bkron \M{O}_u^x \right)
    \left(\V{1}_N \kron \fp_0 \right)
    \\
    = &
    \begin{bmatrix}
        \left(\M{I} \bkron \M{O}_u^x \right) \diag{k = 1 \dots N}{\M[\hat{p}_k][]{R}}
        &
        \left(
            \M{I}
            \bkron
            \left(
                \M{O}_u^x
                \M{I}_R
            \right)
        \right)
        \diag{k = 0...M-1}{ \M[p_{k}][]{R} }\\
    \end{bmatrix}
    \V{X}
    \\
    &
    +
    \left(\M{I} \bkron \M{U}_x^x \right) \cstate_0
    +
    \left(
        \M{I}
        \bkron
        \left(
            \M{O}_u^x
            \V{1}_N
        \right)
    \right)
    \fp_0
    \\
\end{align}
%

Velocity of the \acs{CoM} can be expressed as
%
\begin{equation}
    \cVel =
        \diag{N}{\M{I}_{v}} \cState =
        \diag{N}{\M{I}_{v}} \left( \M{S}\V{X} + \V{s} \right)=
        \M{S}_v \V{X} + \V{s}_v
\end{equation}
%

%%%%%%%%%%%%%%%%%%%%%%%%%%%%%%%%%%%%%%%%%%%%%%%%%%%%%%%%%%%%%%%%%%%%%%%%%%%%%%%%%%%%%%%%%%%%%%%%%%%%%%
\section{Constraints}

%%%%%%%%%%%%%%%%%%%%%%%%%%%%%%%%%%%%%
\subsection{CoP positions}
Simple bounds in the case when supports are rectangular.

\begin{equation}
    \ubarV{z}_k \le \cop[\hat{p}_k]_k \le \barV{z}_k, \quad k = 1 \dots N\\
\end{equation}

\begin{equation}
    \ubarV{Z} \le \CoP \le \barV{Z}
\end{equation}


%%%%%%%%%%%%%%%%%%%%%%%%%%%%%%%%%%%%%
\subsection{Foot positions}
Simple bounds when feasible regions are rectangular.

\begin{equation}
    \ubarV{p}_j \le \fp[p_{j-1}]_j \le \barV{p}_j, \quad j = 1 \dots M
\end{equation}

\begin{equation}
    \ubarV{P} \le \FP \le \barV{P}
\end{equation}

Only initial and final double supports are handled, the respective constraints
can also be represented as simple bounds provided that the feet are aligned.


%%%%%%%%%%%%%%%%%%%%%%%%%%%%%%%%%%%%%%%%%%%%%%%%%%%%%%%%%%%%%%%%%%%%%%%%%%%%%%%%%%%%%%%%%%%%%%%%%%%%%%
\section{Objectives}
Three objectives are minimized: difference between the actual and reference
\acs{CoM} velocity
%
\begin{equation}
    \NORME{\cVel - \cVel_{ref}} = \NORME{\M{S}_v \V{X} + \V{s}_v - \cVel_{ref}},
\end{equation}
%
the \acs{CoP} velocity
%
\begin{equation}
    \NORME{\dCoP} = \NORME{\M{S}_{\dot{z}} \V{X} + \V{s}_{\dot{z}}},
\end{equation}
%
distance between the \acs{CoP} positions and the centers of the feet
%
\begin{equation}
    \NORME{\CoP}.
\end{equation}
%


%%%%%%%%%%%%%%%%%%%%%%%%%%%%%%%%%%%%%%%%%%%%%%%%%%%%%%%%%%%%%%%%%%%%%%%%%%%%%%%%%%%%%%%%%%%%%%%%%%%%%%
\section{QP}
\begin{equation}
\begin{split}
    \MINIMIZE{\CoP, \FP}  & \frac{\alpha}{2} \NORM{\cVel - \cVel_{ref}}^2 +
                              \frac{\beta}{2}  \NORM{\dCoP}^2 +
                              \frac{\gamma}{2} \NORM{\CoP}^2\\
    \SUBJECTTO          & \ubarV{Z} \le \CoP \le \barV{Z} \\
                        & \ubarV{P} \le \FP \le \barV{P}
\end{split}
\end{equation}


%%%%%%%%%%%%%%%%%%%%%%%%%%%%%%%%%%%%%
\subsection{Reference velocity}
\begin{equation}
\begin{split}
    &\NORM{\cVel - \cVel_{ref}}^2
    =
    \NORM{\M{S}_v \V{X} + \V{s}_v   -  \cVel_{ref}}^ 2
    = \\
    &
    \V{X}^T \M{S}_v^T \M{S}_v \V{X}
    +
    \V{X}^T \M{S}_v^T \V{s}_v
    -
    \V{X}^T \M{S}_v^T \cVel_{ref}\\
    &
    +
    \V{s}_v^T \M{S}_v \V{X}
    +
    \cancel{\V{s}_v^T \V{s}_v}
    -
    \cancel{\V{s}_v^T \cVel_{ref}} \\
    &
    -
    \cVel_{ref}^T \M{S}_v \V{X}
    -
    \cancel{\cVel_{ref}^T \V{s}_v}
    +
    \cancel{\cVel_{ref}^T \cVel_{ref}}
\end{split}
\end{equation}

Omitting the  constant terms we obtain
\begin{equation}
    \frac{\alpha}{2} \V{X}^T \M{S}_v^T \M{S}_v \V{X}
    -
    \alpha \cVel_{ref}^T \M{S}_v \V{X}
    +
    \alpha \V{s}^T_v \M{S}_v \V{X}
\end{equation}

%%%%%%%%%%%%%%%%%%%%%%%%%%%%%%%%%%%%%
\subsection{CoP velocity}
\begin{equation}
    \NORM{\dCoP}^2 =
    \NORM{\M{S}_{\dot{z}} \V{X} + \V{s}_{\dot{z}}}^2 =
        \V{X}^T \M{S}_{\dot{z}}^T \M{S}_{\dot{z}} \V{X}
        +
        \V{X}^T \M{S}_{\dot{z}}^T \V{s}_{\dot{z}}
        +
        \V{s}_{\dot{z}}^T \M{S}_{\dot{z}} \V{X}
        +
        \cancel{\V{s}_{\dot{z}}^T \V{s}_{\dot{z}}}
\end{equation}

Omitting the constant terms we obtain
\begin{equation}
    \frac{\beta}{2}
    \V{X}^T \M{S}_{\dot{z}}^T \M{S}_{\dot{z}} \V{X}
    +
    \beta
    \V{s}_{\dot{z}}^T \M{S}_{\dot{z}} \V{X}
\end{equation}

%%%%%%%%%%%%%%%%%%%%%%%%%%%%%%%%%%%%%
\subsection{Displacement from the reference CoP}
\begin{equation}
    \frac{\gamma}{2} \NORM{\CoP}^2
        = \frac{\gamma}{2} \CoP^T \CoP
\end{equation}


%%%%%%%%%%%%%%%%%%%%%%%%%%%%%%%%%%%%%
\subsection{Objective in the matrix form}
\begin{equation}
\begin{split}
    \MINIMIZE{\V{X}}    & \frac{1}{2} \V{X}^T\M{H}\V{X} + \V{h}^T \V{X}
\end{split}
\end{equation}

\begin{equation}
    \M{H} =
        \alpha \M{S}_v^T \M{S}_v
        +
        \beta  \M{S}_{\dot{z}}^T \M{S}_{\dot{z}}
        +
        \gamma
            \begin{bmatrix}
                \M{0} & \M{0} & \M{0}\\
                \M{0} & \M{I} & \M{0}\\
                \M{0} & \M{0} & \M{0}\\
            \end{bmatrix}\\
\end{equation}

\begin{equation}
    \V{h} =
        -
        \alpha \cVel_{ref}^T \M{S}_v
        +
        \alpha \V{s}_v^T \M{S}_v
        +
        \beta  \V{s}_{\dot{z}}^T \M{S}_{\dot{z}}
\end{equation}




%%%%%%%%%%%%%%%%%%%%%%%%%%%%%%%%%%%%%%%%%%%%%%%%%%%%%%%%%%%%%%%%%%%%%%%%%%%%%%%%%%%%%%%%%%%%%%%%%%%%%%
\section{Hierarchical least squares problem}
\begin{description}
    \item[Level 1:]
        \begin{equation}
        \begin{split}
            & \ubarV{Z} \le \CoP \le \barV{Z} \\
            & \ubarV{P} \le \FP \le \barV{P}
        \end{split}
        \end{equation}

    \item[Level 2:]
        \begin{equation}
        \begin{split}
            & \sqrt{\frac{\alpha}{2}} \M{S}_v \V{X} = \sqrt{\frac{\alpha}{2}} \left(\cVel_{ref} - \V{s}_v\right) \\
            & \sqrt{\frac{\beta}{2}} \M{S}_{\dot{z}} \V{X} = \sqrt{\frac{\beta}{2}} \V{s}_{\dot{z}}\\
            & \sqrt{\frac{\gamma}{2}} \CoP = 0\\
        \end{split}
        \end{equation}
\end{description}


%%%%%%%%%%%%%%%%%%%%%%%%%%%%%%%%%%%%%%%%%%%%%%%%%%%%%%%%%%%%%%%%%%%%%%%%%%%%%%%%%%%%%%%%%%%%%%%%%%%%%%
\section{Swing foot trajectory}

%%%%%%%%%%%%%%%%%%%%%%%%%%%%%%%%%%%%%%%%%%%%%%%%%%%%%%%%%%%%%%%%%%%%%%%%%%%%%%%%%%%%%%%%%%%%%%%%%%%%%%
\subsection{Polynomial and boundary conditions}
Trajectory is generated using cubic polynomial of the form
\begin{equation}
    at^3 + bt^2 + ct + d = y_{swing},
\end{equation}
where $t$ is time instance; $a,b,c,d$ are coefficients; and $y_{swing}$ is position of the swing
foot at time $t$.

Derivatives of the cubic polynomial are
\begin{equation}
\begin{split}
    & 3at^2 + 2bt + c = \dot{y}_{swing},\\
    & 6at + 2b = \ddot{y}_{swing},\\
    & 6a = \dddot{y}_{swing}.\\
\end{split}
\end{equation}

There are four boundary conditions for the polynomial, the first two are defined for
the current swing foot state at $t_i = 0$:
\begin{equation}\label{wpg04.eq.swtraj_bc1}
\begin{split}
    & d = y_{swing,i}, \\
    & c = \dot{y}_{swing,i}, \\
\end{split}
\end{equation}
where $y_{swing,i}$ and $\dot{y}_{swing,i}$ are initial position and velocity; the other two
conditions for landing time instance $t_f$ are:
\begin{equation}\label{wpg04.eq.swtraj_bc2}
\begin{split}
    & at_f^3 + bt_f^2 + ct_f + d = y_{swing,f},\\
    & 3at^2 + 2bt + c = \dot{y}_{swing,f},\\
\end{split}
\end{equation}
where $y_{swing,f}$ and $\dot{y}_{swing,f}$ are final position and velocity.
Trajectories along $z$ axis and $x,y$ axes are computed separately. Final position
$y_{swing,f}^z$ for trajectory along $z$ axis is set to the step height during the first
half of the support and to $0$ during the second half:
\begin{equation}
    y_{swing,f}^z =
    \left\{
        \begin{array}{ll}
            h_{step}    & t \le \frac{1}{2}T_{support}; \\
            0           & t > \frac{1}{2}T_{support}. \\
        \end{array}
    \right.
\end{equation}
The final $x,y$ positions are set to the next landing position computed as
\begin{equation}
    \begin{bmatrix}
        y_{swing,f}^x\\
        y_{swing,f}^y\\
    \end{bmatrix}
    =
    \M{V}_{land}\FP,
\end{equation}
where
\begin{equation}
    \M{V}_{land} =
    \begin{bmatrix}
        \M[p_0][]{R} & \M{0}
    \end{bmatrix}.
\end{equation}
Velocity at the end of trajectory is set to zero.


%%%%%%%%%%%%%%%%%%%%%%%%%%%%%%%%%%%%%%%%%%%%%%%%%%%%%%%%%%%%%%%%%%%%%%%%%%%%%%%%%%%%%%%%%%%%%%%%%%%%%%
\subsection{Computation of the desired acceleration}

Whole body control requires current acceleration (at time $t_i$) of the swing foot,
which can be found as
\begin{equation}\label{wpg04.eq.swtraj_acc1}
    2b = \ddot{y}_{swing,i}.
\end{equation}
Hence it is necessary to find coefficient $b$ from equations~\eqref{wpg04.eq.swtraj_bc1}
and~\eqref{wpg04.eq.swtraj_bc2}.

Substitution of~\eqref{wpg04.eq.swtraj_bc1} to~\eqref{wpg04.eq.swtraj_bc2} using $\dot{y}_{swing,f} = 0$
yields
\begin{equation}\label{wpg04.eq.swtraj_bc3}
\begin{split}
    & at_f^3 + bt_f^2 + \dot{y}_{swing,i}t_f + y_{swing,i} = y_{swing,f},\\
    & 3at_f^2 + 2bt_f + \dot{y}_{swing,i} = 0.\\
\end{split}
\end{equation}

Trivial algebraic operations on~\eqref{wpg04.eq.swtraj_bc3} and~\eqref{wpg04.eq.swtraj_acc1}
lead to the following equation:
\begin{equation}\label{wpg04.eq.swtraj_acc2}
    \ddot{y}_{swing,i}
    =
    \frac{6(y_{swing,f} - y_{swing,i})}{t_f^2} - \frac{4\dot{y}_{swing,i}}{t_f}
    =
    \frac{6}{t_f^2}y_{swing,f} -
    \underbrace{
        \frac{6}{t_f^2}y_{swing,i} - \frac{4}{t_f} \dot{y}_{swing,i}
    }_{\mbox{constant}}.
\end{equation}

Consequently
\begin{equation}
    \V{\ddot{y}}_{swing}
    =
    \begin{bmatrix}
        \ddot{y}_{swing,i}^x\\
        \ddot{y}_{swing,i}^y\\
        \ddot{y}_{swing,i}^z\\
    \end{bmatrix}
    =
    \underbrace{
        \frac{6}{t_f^2}
        \begin{bmatrix}
            \M{V}_{land} \\
            \M{0}
        \end{bmatrix}
    }_{\M{V}_{sa}}
    \FP
    +
    \underbrace{
        \frac{6}{t_f^2}
        \begin{bmatrix}
            \M{0} \\
            y_{swing,f}^z
        \end{bmatrix}
        -
        \frac{6}{t_f^2}
        \V{y}_{swing,i}
        -
        \frac{4}{t_f}
        \V{\dot{y}}_{swing,i}
    }_{\V{b}_{sa}},
\end{equation}


%%%%%%%%%%%%%%%%%%%%%%%%%%%%%%%%%%%%%%%%%%%%%%%%%%%%%%%%%%%%%%%%%%%%%%%%%%%%%%%%%%%%%%%%%%%%%%%%%%%%%%
\subsection{Computation of the initial jerk}
Jerk at the beginning of the swing foot trajectory characterizes behaviour of the
polynomial near this point. High values of the jerk indicate rapid change of
acceleration, which in turn increases error in trajectory tracking using piece-wise
constant acceleration. Due to this reason we penalize the jerk at the starting point
of trajectory. Obviously, only $x,y$ components are penalized, since $z$ component
is directly computed.

The jerk can be computed as:
\begin{equation}\label{wpg04.eq.swtraj_jerk}
    \dddot{y}_{swing,i}
    =
    - \frac{12(y_{swing,f} - y_{swing,i})}{t_f^3} + \frac{6\dot{y}_{swing,i}}{t_f^2}
    =
    - \frac{12}{t_f^3}y_{swing,f}
    +
    \underbrace{
        \frac{12}{t_f^3}y_{swing,i} + \frac{6}{t_f} \dot{y}_{swing,i}
    }_{\mbox{constant}}.
\end{equation}

Therefore $x,y$ components of the jerk are expressed as
\begin{equation}
    \V{\dddot{y}}_{swing}^{x,y}
    =
    \begin{bmatrix}
        \dddot{y}_{swing,i}^x\\
        \dddot{y}_{swing,i}^y\\
    \end{bmatrix}
    =
    \underbrace{
        -
        \frac{12}{t_f^3}
        \M{V}_{land}
    }_{\M{V}_{sj}}
    \FP
    +
    \underbrace{
        \frac{12}{t_f^3}
        \V{y}_{swing,i}
        +
        \frac{6}{t_f^2}
        \V{\dot{y}}_{swing,i}
    }_{\V{b}_{sj}}
    =
    \underbrace{
        \begin{bmatrix}
            \M{0}   &   \M{0}   &   \M{V}_{sj}
        \end{bmatrix}
    }_{\M{A}_{sj}}
    \V{X}_{mpc}
    +
    \V{b}_{sj}
    .
\end{equation}
\input{../../../doc/latex/Styles/wpg_notation_undefine.tex}
